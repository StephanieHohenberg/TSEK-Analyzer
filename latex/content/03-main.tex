\chapter{Vorgehen}
\label{chapter:vorgehen}
Im Folgenden wird das Vorgehen erläutert um die im ersten Kapitel genannten Ziele zu erreichen. Dabei wurde folgende sechs Aktivitäten ausgearbeitet, die nicht auf einander aufbauen, sondern in einem iterativ Prozess miteinander verzahnt sind, welcher in der Grafik \ref{img:process} veranschaulicht wurden. \\

\begin{itemize}
    \item Sichtung der Publikationen
    \item Identifizierung von Kontextfaktoren
    \item Klassifizierung von Kontextfaktoren
    \item Charaktierisierung von Publikationen
    \item Dokumentierung und Erstellung von Artefakten
    \item Analyse der Artefakte
    \item Anfertigung einer Taxonomie
\end{itemize}

\section{Sichtung der Publikationen}
Zur Identifikation von Kontextfaktoren, werden die Publikationen der technischen Artikel veröffentlicht in der ISCE 2020, in mehreren Iterationen gesichtet. \\

In der 1. Iteration basiert die Reihenfolge, anhand welcher die Publikationen gelesen werden, auf die Auszeichnungen und Badges, die die Publikationen erhalten haben, da angenommen wird, dass diese Publikationen eine höhere Qualität aufweisen. \\
Von den 129 technischen Arbeiten der ISCE2020 haben zehn Publikationen die Auszeichnung \textit{"ACM SIGSOFT Distinguished Paper Award"} und drei Publikationen die Auszeichnung \textit{"ACM SIGSOFT Distinguished Artifact Award"} verliehen bekommen. Außerdem wurden 34 Publikationen mit dem Badge \textit{"ACM Artifacts Evaluated Reusable"} und 47 Publikationen mit dem Badge \textit{"ACM Artifacts Available"} ausgezeichnet. Dabei besteht jedoch eine Überschneidung von diesen Publikationen, wie in der Grafik \ref{img:venndiagramm} dargestellt. Insgesamt haben 64 von den 129 Publikationen eine Auszeichnung und/oder ein Badge. \\
Zuerst werden die 13 Publikationen mit einer Auszeichnung gelesen, danach die 30 Publikationen mit zwei Badges, die 21 Publikationen mit einem Badge und im Anschluss die restlichen Publikationen. Dabei ist die Reihenfolge innerhalb dieser vier Gruppen zufällig gewählt. \\

Während dem erstmaligen Sichten einer Publikation, soll diese charakterisiert werden, welches eine weitere Aktivität des Vorgehens darstellt und in der Sektion \ref{sec:character} näher beschrieben wird.\\

Unter anderem wird abhängig von dieser Charakterisierung, aber auch abhängig von den in der 1.Iteration bereits identifizierten Kontexten und von den Themengebiet einer Publikation, die Publikationen gruppiert. Innerhalb dieser Gruppen werden die Publikationen in einer 2.Iteration erneut gesichtet und kritischer analysiert. \\


\section{Identifizierung von Kontextfaktoren}
Da der Aufwand eine Publikation komplett in voller Länger intensiv zu lesen nicht dem Nutzen alle Kontextfaktoren zu finden enspricht und angenommen wird das Kontextfaktoren unterschiedlich häufig an unterschiedlichen Stellen einer Publikation vorkommen, wird sich beim Sichten der Publikationen auf diese Stellen konzentriert. \\
Dazu wird im Folgenden weitere Annahmen getroffen zu den Stellen, an welchen Kontextfaktoren auftreten. \\

Es wird angenommen, dass Kontextfaktoren abhängig von dessen Zwecken an unterschiedlichen Stellen in einer konkreten Publikation vorkommen. \\
Denn Kontextfaktoren haben nicht nur den Zweck die Verallgemeinbarkeit einer Publikation zu beschreiben, wie in Kapitel 1 bei der Zielsetzung ausführlich erläutert, sondern auch zu anderen Zwecken. Denn es kann, wie bereits erwähnt, unterschieden werden zwischen den Kontext in den eine empirische Studie durchgeführt wurde und die Kontexte auf welche dessen Ergebnisse verallgemeinbar sind. \\
Dieser Zusammenhang zwischen dem Zweck eines Kontextfaktors und der Stelle in der Publikation wird im Weiteren erläutert und in der Tabelle \ref{table:vorkommen} aufgeführt. \\

In dem Titel und in der Einleitung einer Publikation ist zu erwarten, dass Kontextfaktoren zur Charakterisierung des Themenbereich oder zur Definition des untersuchten Problems verwendet werden. \\
Des Weiteren können Kontextfaktoren bei der Beschreibung einer Technik oder des Vorgehens einer empirischen Studie aufgeführt werden um Einzelheiten zu Beleuchten und zu Beschreiben in welchen Kontext Daten erhoben wurden. \\
Bei der Diskussion verwandter Arbeiten, haben Kontextfaktoren den Zweck die konkrete Publikation einzugrenzen und von anderen Publikationen abzugrenzen. An dieser Stelle können Kontextfaktoren genannt werden, die das Thema der Publikation einschränken und somit spezialisieren. Es kann beschrieben werden, in welchen Kontext die Studie nicht durchgeführt wurde und außerdem kann die Studie oder dessen Ergebnisse sich in Betrachtung anderer Arbeiten positioniert werden. \\
Bei den Schlussforgerungen, Nennung von Forschungsbeiträgen und bei der Diskussion zur Verallgemeinbarkeit wird unterem Anderem diskutiert, was die Validität der Publikation gefährden könnte und Kontextfaktoren diesbezüglich aufgestellt. Es ist zu Erwarten, dass an dieser Stelle viele Kontextfaktoren zu finden sind, die sich konkret auf die Verallgemeinbarkeit der Arbeit beziehen. \\

\begin{table}[h!]
\begin{tabular}{ r | l }
 Zweck & Vorkommen \\
  \hline
  Thema & Titel, Einleitung, Recherchefragen \\
  Detail & Beschreibung, Vorgehen, Datenerhebung \\
  Abgrenzung & Diskussion verwandter Arbeiten \\
  Verallgemeinbarkeit & Schlussfolgerungen, Forschungsbeiträge, Diskussion\\
\end{tabular}
\caption{Vorkommen von Kontextfaktoren nach Zweck}
\label{table:vorkommen}
\end{table}




Beim Sichten der Publikationen der ISCE 2020, wird sich auf die in Tabelle  \ref{table:vorkommen} aufgelisteten Stellen beschränkt. \\
Um sich einen Überblick über eine Publikation zu schaffen, werden zuerst die Titel der einzelnen Sektionen einer konkreten Publikation gelesen und zugeordnet, zu welchen Zweck Kontextfaktoren in einer konkreten Sektion vorkommen könnten. Anschließend werden die Sektionen, von denen vermutet wird, dass dort Kontextfaktoren vorkommen, in folgender Reihenfolge abhängig des Zweckes gelesen: Thema, Abgrenzung, Verallgemeinbarkeit, Detail. \\
Zunächst werden Sektionen, die sich auf das Thema der Publikation beziehen gelesen, um nicht nur Kontextfaktoren bezüglich des Themas zu finden, sondern auch um eine schnelle Einleitung in das Thema der Publikation zu erlangen. Anschließend werden die Sektionen bezüglich der Abgrenzung gesichtet, wie die Diskussion verwandter Arbeiten, um das Thema einzugrenzen. Danach werden die Sektionen bezüglich der Verallgemeinbarkeit, wie die Schlussfolgerungen und Diskussion gelesen. Dabei wird analysiert, welchen Beitrag die Publikation leistet. Im Anschluss werden die Sektionen der Publikation angeschaut, die Details erläutern.  \\

Bei dieser Lesetechnik, werden die Publikationen mit dem Fokus auf aller vier Zwecke nacheinander gesichtet. Das hat den Vorteil, dass es zu keinem Kontextwechsel kommt, wie es bei der Alternative der Fall wäre, wenn in vier Iterationen die Publikationen lediglich hinsichtlich eines Zweckes gesichtet werden, sprich zuerst werden alle Publikationen hinsichtlich des Themas gelesen, danach alle hinsichtlich der Verallgemeinerung, und so weiter. Dadurch könnte das Verständnis einer konkreten Publikation gemindert werden, sodass implizite Kontextfaktoren nicht entdeckt werden könnten. Jedoch hätte die Alternative den Vorteil, dass Beziehungen zwischen Kontextfaktoren verschiedener Publikationen bezüglich eines Zweckes besser erkannt werden könnten. \\

Da durch die Lesetechnik eventuell nicht alle Kontextfaktoren gefunden werden, wird in Kapitel \ref{chapter:evaluation} die Lesetechnik evaluiert und abgeschätzt, wie viele Kontextfaktoren unerkannt blieben als \textit{false negative}. Eine maximale bzw. vollständige Abdeckung ist jedoch nicht das Ziel dieser Arbeit. \\

Eine weitere Schwierigkeit ergibt sich dadurch, dass lediglich die Autorin dieser Masterarbeit alle Publikationen sichtet, Kontextfaktoren identifiziert und klassifiziert. Somit besteht zwar ein Grad an Einheitlichkeit. Jedoch könnten Verzerrungen und Bias entstehen. \\
Dies könnte verhindert werden, wenn mehrere Autoren und Autorinnen beteiligt wären und diese Aktivität unabhängig voneinander ohne gegenseitige Beeinflussung für eine zufällig ausgewählte Teilmenge der Publikation durchführen würden. Wenn sich dann noch die Teilmengen überschneiden, sodass jede Publikation mehrmals gesichtet wird, hat es den Vorteil, dass nicht nur mehr Kontextfaktoren identifiziert werden könnten, sondern auch, dass die interne Validität der Arbeit verstärkt wird. \\
Da jedoch nach der eidesstaatlichen Erklärung diese Arbeit komplett alleine durchgeführt wurde, ist dieser Ansatz nicht möglich. Allerdings wäre es vorstellbar, dass eine Folgearbeit geleistet werden kann, um Ergebnisse dieser Arbeit zu evaluieren und die interne Validität dieser Arbeit zu stärken. \\

%% TODO alternativer Lösungansatz? Wie können Verzerrungen und Bias meinerseit vermieden werden? \\

\section{Klassifizierung von Kontextfaktoren}
\label{sec:class-context}
Im Anschluss der Sichtung einer konkreten Publikation werden die identifizierten Kontextfaktoren dieser Publikation klassifiziert und Faktorengruppen werden gebildet. Dabei wird ein Oberbegriff gefunden, der den Kontextfaktor beschreibt. Ein Beispiel dafür wäre, wenn eine Publikation \footnote{z.B. in der Publikation XXX} Softwareprojekte untersucht, die durch das Kriterium \textit{Nutzung von Java} ausgewählt wurden, lässt sich der Kontextfaktor \textit{Java} erkennen und ein passender Oberbegriff dafür, wäre \textit{Technologien}, welche die Faktorengruppe für alle Kontextfaktoren die Technologien beschreiben bildet. \\

Da bei den Publikationen die zuerst gesichtet werden, noch keine Klassen bzw. Faktorengruppen bestehen, weil die Faktorengruppen fortschreitend nach jeder Publikation ausgearbeitet werden, müssen die entstandenen Faktorengruppen und dessen zugeordneten Kontextfaktoren regelmäßig evaluiert und angepasst werden. Da alle Kontextfaktoren, die bereits zu einer Faktorengruppe zugeordnet sind, sich mit dem neuklassifizierten Kontextfaktor einer anderen Publikation ähneln müssen und die Faktorengruppe zu allen zugeordneten Kontextfaktoren passen muss. \\
Dabei kann es vorkommen, dass eine Faktorengruppe umbenannt wird auf einen mehr generalisierten Begriff, um alle Kontextfaktoren zu umfassen, oder dass eine Faktorengruppe auf zwei spezialisierte Faktorengruppen aufgeteilt werden, sodass eine Hierarchie von Faktorengruppen entstehen könnte. \\

%% TODO Lerneffekt EMPI \\

%% TODO \\
Wie bereits in Kapitel \ref{chapter:background} Sektion \ref{sec:verwandt} erwähnt, sollen ebenfalls verwandte Arbeiten evaluiert werden, um Kontextfaktoren und Faktorenklassen zu finden. Außerdem können geeignete Benennungen für Faktorengruppen  übernommen werden.


\section{Charakterisierung von Publikationen}
\label{sec:character}

Bei dem Leseprozess, sollen nicht nur Kontexte einer Publikation identifiziert werden. Die Publikation soll auch charakterisiert werden, damit Korrelationen zwischen Kontextfaktoren und den Publikationen erkannt werden können. \\

Im Gegensatz zu der Klassifizierung von Kontextfaktoren in Faktorengruppen, dessen Anzahl im Leseprozess steigt, werden die Charakterisierungskriterien und Optionen nicht fortschreitend auserarbeitend, sondern vorab definiert. \\
Da bereits in Kapitel \ref{chapter:background} \textit{Hintegrund} Sektion \ref{sec:empi} erläutert wurde, wie sich empirische Arbeiten unterscheiden können, wird hier nicht mehr näher dadrauf eingagen, sondern auf diese Sektion verwiesen. Anhand der Unterscheidungsfaktion von empirischer Arbeiten wurden Charakterisierungskriterien ausgewählt, welche in der Tabelle \ref{table:character} zusammengefast wurden. \\

%% TODO Beschreibungen statt Tabelle, später für Statistik Tabelle \\
\begin{table}[h!]
\begin{tabular}{ r | l }
 Kriterium & Optionen \\ 
  \hline
  resultierender Beitrag der Publikation & XXX \\  
  empirische Methoden & Umfrage, ... XXX ref auf Grafik? \\
  Art & quantitative und/oder qualitative \\
\end{tabular}
\caption{Kriterien zur Charakterisierung von Publikationen}
\label{table:character}
\end{table}




\section{Dokumentierung und Erstellung von Artefakten}
Um diesen iterativen Prozess, welcher in Grafik \ref{img:process} dargestellt ist, zu bewerkstelligen und eine Übersicht von Kontextfaktoren und Faktorengruppen zu ermöglichen und eventuell bereits eine Hierarchie von Faktorengruppe abbilden zu können, wird eine Datenbank angelegt. \\

%% TODO \\


Diese Datenbank wurde durch das kostenlose Datenbank Management System \textit{Open Office Base} von \textit{Apache} erstellt und ist in odp.-Format zugänglich \footnote{Link zur Datenbank XX}. Es wurde sich für dieses Werkzeug entschieden, da es kostenlos verfügbar ist und die Autorin mit diesem vertraut ist.\\

Die Datenbank enthält vier verschiedene Objekte, desse Attribute und Beziehungen zueinander im Folgendem erläutert werden.
\begin{itemize}
    \item Kontextfaktor
    \item Faktorengruppe
    \item Publikation
    \item Change-Log
\end{itemize}

Jeder \textbf{Kontextfaktor} steht in Beziehung mit einer Publikation, die diesen Faktor erwähnt. Dabei wird die Stelle, wo der Faktor vorkommt durch die Seitenzahl, den Zweck und den Paragraphentitel und -typ in entsprech, siehe Tabelle \ref{table:vorkommen}. 
Des Weiteren wird in der Datenbank ein Zitat aufgenommen, um den Prozess der Evaluierung von Kontextfaktoren zugeordnet zu einer Faktorengruppe zu beschleunigen. \\ 

Des Weiteren steht ein Kontextfaktor in Verbindung zu nur einer Faktorengruppe, wohingegen eine Faktorengruppe Verbindungen zu mehreren Kontextfaktoren hat, da eine Faktorengruppe mehrere Kontextfaktoren gruppiert. \\

Die \textbf{Faktorengruppe} enthält lediglich zwei Attribute und zwar die Benamung und eine kurze Beschreibung der Gruppe. Dies soll ebenfalls den Evaluierungsprozess von Zuordnungen beschleunigen. 
Wenn dabei eine Faktorengruppe generalisiert wird, wird der Name und die Beschreibung angepasst. Wenn eine Faktorengruppe in zwei spezialisierte Faktorengruppen aufgeteilt werden soll, weil die Faktorengruppe zu viele verschiedene Kontextfaktoren enthält, bleibt das Objekt dieser Faktorengruppe erhalten. Jedoch werden Verbindungen zu den Kontextfaktoren entfernt. Diese werden zu mindesten zwei neuen, spezialisierteren Faktorengruppen aufgeteilt, welche eine Verbindung zu der ehemals zugeordneten Faktorengruppe, als \textit{Parent-Faktorengruppe} haben. \\

Für jede \textbf{Publikation} wird ein Objekt angelegt, sobald der Sichtungsprozess für eine konkrete Publikation angefangen wird. Dabei werden zuerst Attribute definiert, die bereits bekannt sind und die keinen Analyse Aufwand erfordern, wie der Titel der Publikation, die in der Publikation aufgelisteten Keywords und das von ISCE zugeordnete Genre. Des Weiteren stellt der pdf-Name der Publikation den Identifikator des Objektes dar. \\ %% Anzahl Genres? \\
Nachdem eine Publikation zu Ende gesichtet wurde und die Identifizierung von Kontextfaktoren innerhalb dieser Publikation abgeschlossen ist, wird die Charakterisierung der Publikation, wie in Sektion \ref{sec:character} beschrieben, durchgeführt und entsprechende Attribute am entsprenden Datenbank Objekt werden angepasst. \\

Diese Verbindungen zwischen den Objekten und dessen Attribute ist in der Graphik \ref{img:datamodell} dargestellt, welche das Datenmodell des Datenbank Managemt Systems darstellt. \\

Da wie in der vorherigen Sektion \ref{sec:class-context} beschrieben, müssen die Faktorengruppen regelmäßigen im Sichtungsprozess, während der Klassifizierung von Kontextfaktoren evaluiert und eventuell angepasst werden. Um diesen Prozess der Evaluierung von Zuordnungen von Kontextfaktoren zu Faktorengruppen und um die gemachten Anpassungen zu dokumentieren, besonders hinsichtlich der Generalisierung und Aufteilung von Faktorengruppen, wird ein weiteres Objekt gepflegt: \textbf{Change-Log}. \\
Dadurch kann nicht nur die Arbeitsweise und die angestellten Überlegungen besser nachvollzogen werden, sondern der ganze Prozess der Klassifizierung von Kontextfaktorengruppen kann besser rekonstruiert, besser evaluiert und besser bewertet werden. Des Weiteren wird dadurch eine Historie erstellt, sodass die Evolution von Faktorengruppen sichtbar wird und sodass Arbeitsschritte besser reflektiert und gegebenfalls rückgängig gemacht werden können. \\

Ein Change-Log Objekt soll protokollieren, wann sich die Zuordnung von Kontextfaktoren zu einer Faktorengruppe verändert, hinsichtlich der Umbennung einer Faktorengruppe oder der Aufteilung einer Faktorengruppe. Dabei wird bei der Erstellung des Objektes zwei Verbindungen definiert, zu der betroffenen Faktorengruppe und zu dem Kontextfaktor, welcher diese Anpassung der Faktorengruppe ausgelöst hat. \\
Das Objekt hat neben den zwei Verbindungen drei Attribute. Es wir ein Attribut zum Typen des Veränderung der Faktorengruppe definiert, welches den Wert Umbenennung oder Aufteilung haben kann. Des Weiteren wird die Anzahl der Kontextfaktoren festgehalten, die vor der Umbenennung bzw. vor der Aufteilung der Faktorengruppe, zusammengefasst waren. Bei der Umbennung einer Faktorengruppe kann außerdem die vorherige Name der Gruppe dokumentiert werden. \\
 
Es wurde auch abgewägt, ob zwei Zeitstempeln in den anderen Objekten eingeführt werden sollen, um zu dokumentieren, wann ein Kontextfaktor, eine Faktorgruppe oder eine Publikation erstellt wurde und zuletzt aktualisiert wurde. \\
Es ist jedoch anzunehmen, dass Veränderungen und Anpassungen überwiegend in den Faktorengruppen vorgenommen werden, da diese in Prozess der Klassifizierung von Kontextfaktoren regelmäßig evaluiert werden müssen. Da die Veränderungen der Faktorengruppen durch Change-Log Objekte dokumentiert werden, sind daher keine Zeitstempel fuer Faktorengruppen notwendig.\\
Bei Publikation und Kontextfaktoren hingegen ist anzunehmen, dass diese nach dessen Erstellung lediglich durch Verständnisverbesserungen aktualisiert werden.  \\
Desweiteren sollte auch die Reihenfolge in der Publikationen gesichtet und Kontextfaktoren identifiziert werden, und somit entsprechende Objekte in der Datenbank erstellt werden, nicht von Belang sein, da die Reihenfolge in der Publikation gelesen wird, zufällig ist.\\

%% TODO UPDATE MaxQDA \\
%% TODO Verbindung um Kette von Kontextfaktoren innerhalb einer Publikation darzustellen? \\
%% TODO Kontextfaktor wird an vers Stellen der Publikation mehrmals erwähnt zu verschieden Zwecken und in vers Tiefe/Detailreiche/ Charakterisierung \\


\section{Analyse der Artefakte}
\label{sec:analysis-step}
Im nächsten Kapitel \ref{chapter:impl} \textit{Durchführung} wird nicht nur die bereits beschriebene Lesetechnik angewandet und das bereits erläuterte Vorgehen durchgeführt, sondern auch kritisch analysiert. Diese Analyse soll anhand folgende Recherchefragen geleitet werden, welche neben den bereits erläuterten Vorgen in Kapitel \ref{chapter:evaluation} \textit{Evaluation} evaluiert werden und somit beantwortet werden. \\

\paragraph{Recherchefrage 1:} Welche Kontextfaktoren werden in den Publikationen erwähnt? In wie weit beschreiben Autoren und Autorinnen den Kontext ihrer Publikationen? In welcher Form kommen Kontextfaktoren vor? Sind diese explizit oder implizit erkennbar? Sind andere Zwecke als, die in der Tabelle \ref{table:vorkommen} genannten, erkennbar? 

\paragraph{Recherchefrage 2:} Unterscheiden sich die Kontexte und Kontextfaktoren abhängig von den Arten der Publikationen und von der Vorgehensweise der empirischen Studie?

\paragraph{Recherchefrage 3:} In wie weit diskutieren Autoren und Autorinnen über die Verallgemeinbarkeit ihr Publikationen, Ergebnisse und Techniken? Bestehen Lücken in den Diskussionen zur Verallgemeinbarkeit? Wie unterscheiden sich die Kontextfaktoren zum Zweck der Verallgemeinbarkeit zu den Kontextfaktoren anderer Zwecke?
%% TODO Threat of Validity, sollte viel drin stehen, aber oft nur Blabla \\

Bei der kritischen Analyse soll vor Allem das erstellte Datenbank Management System genutzt werden um die resultierende Artefakte des erläuterten Vorgehens bestehend aus den Datenbank Objekten und dessen Verbindungen zu analysieren. Dazu werden geeignete SQL-Abfragen erstellt um relevante Daten für die Beantwortung der Recherchefragen zugänglich zu machen und auswerten zu können. Des Weiteren werden Korellationen und Zusammenhänge in geeigneten Grafiken visualisiert oder in Tabellen zusammengefasst.

%% TODO ISCE, Hohe Qualitaet, aber nicht immer relevant \\
%% TODO Bewertung der Existenzberechtigung eines Themas, starke/schwache Relevanzbegruendung? <-> Verallgemeinbarkeit \\

\section{Anfertigung einer Taxonomie}
Im Anschluss des bereits erläuterten Vorgehen werden ersten Schritte zu einer Taxonomie von Software Enginnering Kontexten gemacht. Dabei soll wie bei dem Schritt zur \textit{Analyse der Artefakten}, welcher in Sektion \ref{sec:analysis-step} dieses Kapitels erläutert wurde, das Datenbank Management System genutzt werden, wo bereits durch die Verbindungen zwischen Faktorengruppen eine Hierarchie erkennbar sein kann. \\

Jedoch besteht ein Kontext nicht nur aus einer Faktorengruppe, sondern aus mehreren. Daher soll analysiert werden, in welchen Kombinationen Faktorengruppe in den Publikationen vorkommen und in wie weit diese Kombinationen sich häufen. Dabei soll nicht nur der Kontext erkannt werden, in der eine empirische Arbeit durchgeführt wurde, sondern auch die Kontexte auf welche diese verallgemeinbar ist. \\
Wie bei dem Schritt zur \textit{Analyse der Artefakten}, welcher in Sektion \ref{sec:analysis-step} dieses Kapitels erläutert wurde, sollen geeignete SQL-Abfragen geschrieben werden, um die Kontexte aufzudecken.

Ebenfalls soll, wie im Schritt zur \textit{Klassifizierung von Kontextfaktoren}, welcher in Sektion \ref{sec:class-context} dieses Kapitels beschrieben wurde, verwandte Arbeiten genutzt werden, um die Kontexte und die Taxonomie zu evaluieren oder zu ergänzen.

%% TODO \section{} - Zusammenfassung, Beschreiben des Vorgehens \\
%% TODO Begriffe hinterfragen, zb. Commits, LoC – Gewicht? \\
%% TODO Öfteres Kodieren, semantische Konzepte sammeln \\



%%% Netzwerk zwischen Publikation -> durch Zurodnung von Kontexten?  
%% Welche Publikation/Ergebnisse ist auf eine andere Publikation anwendbar/übertragbar? 
%% Welche und wie viele Publikationen werden im gleichen Kontext geschrieben?

%% TODO Präsenz und Vergangenheitsformen abchecken - gehören Teile dieses Kapitels nicht eher in der Implementierung oder woanders hin? oder passen sie hier doch besser? Geschmacksache? \\

% Chapter 3
% - Gewählter Lösungsansatz, Alternativen, Abwägungen
