\chapter{Einführung}
\label{chapter:intro}
Im Folgendem wird die Motivation zur Anfertigung dieser Masterarbeit "\textit{Erste Schritte zu einer Taxonomie von Software Engineering Kontexten}" erläutert, Ziele werden gesetzt und die Struktur, die der Masterarbeit zu Grunde liegt, wird beschrieben.

\section{Motivation}
Die Qualität von Arbeiten in der Forschung von Software Engineering werden durch verschiedener Faktoren gemessen. Ein Qualitätsfaktor ist die Relevanz einer Arbeit.
Dabei spielt die Verallgemeinbarkeit eine Rolle: die Anwendbarkeit der vorgestellten Techniken bzw. die Generalisierbarkeit und die Übertragbarkeit der empirischen Ergebnisse auf andere Kontexte. \\

Es kann unterschieden werden zwischen den Kontext in welchem einem empirische Studie durchgeführt wurde und die Kontexte auf welche dessen Ergebnisse sich verallgemeinern lassen und welche somit nicht nur die Relevanz der Arbeit in dem Feld Software Engineering beschreiben, sondern auch, ob die Arbeit und dessen Ergebnisse relevant für andere Arbeiten sind und für deren Beweisführung genutzt werden kann. \\

Jedoch fehlt in der Forschung von Software Engineering eine einheitliche Terminologie, um den Kontext einer Arbeit knapp beschreiben zu können. \\
Wenn der Kontext allerdings gar nicht erläutert wird, nur mit wenig Details und Sorgfalt, kann zum Einen der Kontext einer Arbeit, sowie dessen Verallgemeinbarkeit und Relevanz weniger gut eingeschätzt werden und somit können Verfasser und Verfasserinnen von anderen empirischen Arbeiten schwieriger beurteilen, ob die Ergebnisse einer Arbeit relevant für ihre eigenen Arbeiten sind. Zu Anderem können dadurch nur Annahmen über den Kontext getroffen werden. \\
Dies kann zur Folge haben, dass Arbeiten, die auf Ergebnisse anderer Arbeiten basieren, fehlgeleitet werden, sich fälschlicherweise an Ergebnisse anderer Arbeiten stützen und dieser fälschlicherweise zur Beweisführung nutzen. \\ 

Damit der Kontext und somit die Relevanz von Arbeiten besser beschrieben und besser eingeschätzt werden kann und damit die Forschung von Software Engineering Beweis-orientierter werden kann, soll in dieser Arbeit erste Schritte zu einer Taxonomie von Software Enginnering Kontexten getan werden und eine Terminologie geschaffen werden.

\section{Ziele}
Das Ziel dieser Arbeit ist es Kontextfaktoren zu finden, welche sich auf die Anwendbarkeit der vorgestellten Technik oder auf die Generalisierbarkeit und Übertragbarkeit der empirischen Ergebnisse beziehen und welche somit den Kontext einer Arbeit beschreiben auf dessen diese verallgemeinbar ist.
%Anwendbarkeit ist kein binäres Kriterium, sondern ein graduelles in unscharfer Logik (fuzzy logic). Der konkrete Wert ist in der Regel nicht bekannt.

Um das Ziel zu erreichen, werden in dieser Arbeit zahlreiche Publikationen gesichtet, welche innerhalb der Konferenz \textit{International Conference on Software Engineering}, im Weiteren abgekürzt als ICSE, veröffentlicht wurden. Dabei wird sich nicht auf die Konferenz eingeschränkt, sondern auch auf das Veranstaltungsjahr der Konferenz und auf den Typ der Publikationen. Es werden nur technische Arbeiten von dem letztem Jahr 2020 der ISCE untersucht, um Kontextfaktoren aus der aktuellsten Software Engineering Literatur zu ermitteln. Es wurde sich für die ISCE Konferenz entschieden, da die Artikel, die in Rahmen dieser Konferenz veröffentlicht und vorgestellt werden, eine hohe Qualität aufweisen und somit erwartet wird, dass in den Artikeln die Kontexte beschrieben werden. \\

Bei der Untersuchung der Publikationen, ist nicht nur die Identifizierung der Kontextfaktoren ein Ziel, sondern auch dessen Klassifizierung und eine kritische Analyse. \\
Kontextfaktoren werden in thematischen Faktorenklassen zusammengefasst und Korrelationen zwischen den Kontextfaktoren bzw. Faktorenklassen untersucht. \\
Welche Kontextfaktoren bzw. Faktorenklassen tauchen oft bzw. typischerweise zusammen auf? Welche Kombinationen sind wichtig? Sind alle Faktorenklassen und Kombinationen gleich relevant? Aus welchen Kombinationen von Kontextfaktoren und Faktorenklassen lässt sich ein Kontext beschreiben, welcher die Anwendbarkeit einer Arbeit umfasst? \\
% Begriffe hinterfragen, zb. Commits, LoC – Gewicht? 
%Öfteres Kodieren, semantische Konzepte sammeln

Aus den Kontexten, die in den Publikationen der ISCE 2020 vorkommen, soll eine Taxonomie erstellt werden und somit eine Terminologie geschaffen werden. Dafür ist es notwenig die Kontexte zu benennen, falls kein gängiger Name vorhanden ist. \\

Des Weiteren werden die Korrelationen zwischen Faktorenklassen und den Recherchetyp betrachtet. Da zu erwarten ist, dass sich die Kontextfaktoren innerhalb der verschiedenen Recherchetypen unterscheiden. \\
Außerdem ist zu erwarten, dass Kontextfaktoren an verschiedenen Stellen in einer Arbeit Erwähnung finden. Daher wird auch untersucht, wo Kontextfaktoren in Publikationen beschrieben werden und wie gründlich das getan wird, oder wo Kontextfaktoren nur implizit genannt werden. \\



\section{Struktur der Masterarbeit}
Die Masterarbeit besteht aus sieben Kapitel. \\

Zunächst wird in \textbf{Kapitel \ref{chapter:background} \textit{Hintergrund}}, Wissen zu den grundlegenden Themen \textit{Qualitätsfaktoren empirischer Arbeiten} und \textit{Unterscheidungsfaktoren von empirischen Arbeiten} vermittelt. Des Weiteren werden \textit{verwandte Arbeiten} vorgestellt. \\

In \textbf{Kapitel \ref{chapter:vorgehen} \textit{Vorgehen}} werden die Aktivitäten beschrieben, um die gesetzten Ziele zu erfüllen: \textit{Identifizierung von Kontextfaktoren}, \textit{Klassifizierung von Kontextfaktoren}, \textit{Charakterisierung von Publikationen}, \textit{Dokumentierung und Erstellung von Artefakten}, \textit{Analyse der Artefakte} und \textit{Anfertigung einer Taxonomie}. \\
Dabei wird nicht nur erklärt, wo Kontextfaktoren vorkommen könnten, sondern auch die Lesetechnik wird vorgestellt, Recherchefragen werden aufgestellt und die Verwendung von Werkzeugen wird erläutert. \\

In darauf folgendem \textbf{Kapitel \ref{chapter:impl} \textit{Durchführung}} wird die Realisierung des Vorgehens und die entstandenen Artefakte in Form von \textit{Kontextfaktoren und Faktorengruppen}, sowie in Form von \textit{Kontexte und einer Taxonomie} dokumentiert. \\

Anschliessend wird in \textbf{Kapitel \ref{chapter:evaluation} \textit{Evaluation}} \textit{Schlussfolgerungen} aus diesen Artefakten gezogen und die zuvor in Sektion \ref{sec:analysis-step} aufgestellten \textit{Recherchefragen} werden beantwortet. Des Weiteren werden die \textit{Kontexte}, die \textit{Taxonomie}, sowie die \textit{Lesetechnik} evaluiert. \\

In \textbf{Kapitel \ref{chapter:results} \textit{Ergebnisse}} werden die Ergebnisse dieser Arbeit zusammengefasst und abgegrenzt. \\

Im letzten \textbf{Kapitel \ref{chapter:conclusion} \textit{Abschluss}} erfolgt eine \textit{Diskussion der Ergbnisse} und ein \textit{Ausblick}. \\
        