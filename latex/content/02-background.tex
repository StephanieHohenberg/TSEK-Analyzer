\chapter{Hintergrund}
\label{chapter:background}
Im Folgendem wird der Hintergrund der Masterarbeit vorgestellt. Dazu wird nötiges Hintergrundwissen zu den Qualitätsfaktoren empirischer Arbeiten und den Arten von empirischen Arbeiten bereitgestellt und verwandte Arbeiten werden präsentiert.\\

\section{Definitionen}

\subsection{Qualitätsfaktoren von empirischen Arbeiten}
Wie bereits in Kapitel \ref{chapter:intro} erwähnt ist die Relevanz einer Arbeit ein Qualitätsfaktor um empirischen Arbeiten zu bewerten. Jedoch wird in der empirischen Berwertung im Software Enginnering auch über andere Qualitätsfaktoren diskutiert. \\

Glaubwürdigkeit kann eine Publikation erlangen durch die Gründlichkeit der Beschreibungen, Offenheit zu anderen Ergebnissen, Beschreibung von Einschränkungen, klare und anschauliche Beschreibung der Ergebnisse und der Motivation. \\
%% TODO interne Validität, keine Bias, Tracking der Schlussfolgerungen \\

Relevanz erreicht eine Publikation, wenn das Thema interessant und wichtig ist. Es ist nicht überspezialisiert, sondern ist übertragbar und generalisierbar. \\
%% TODO externe Validität,  \\

%% TODO Zusammenspiel zwischen externe und interne Validität \\

%% TODO EMPI, andere Quellen \\


\subsection{Unterscheidungsfaktoren von empirischen Arbeiten}
\label{sec:empi}
Die Publikation der ISCE2020 stellen zu XX\% eine neue Technik vor, die zu XX\% mit empirischen Studien evaluiert wird. Die anderen XX\% der Publikationen sind durch Recherchefragen motiviert und liefern durch empirische Studien ebenfalls empirische Ergebnisse. \\

Da innerhalb dieser Arbeit zahlreiche Publikationen gesichtet werden und auf die Anwendbarkeit der vorgestellten Techniken oder Übertragbarkeit der empirischen Ergebnisse betrachtet werden, werden in dieser Sektion die verschiedenen Arten empirischer Arbeiten vorgestellt und wie empirische Arbeiten unterschieden werden können. \\

Zu Einem lassen Arbeiten sich einordnen anhand der Art resultierender Beiträge, aber auch anhand dessen Vorgehenweise, sowie ihrer Anwendung von empirischer Methoden. \\

%% TODO quantitaiv, qualitativ  \\

%% TODO EMPI, UCD, andere Quellen  \\
%% TODO Grafik \\


\section{Verwandte Arbeiten}
\label{sec:verwandt}

Im Folgendem wird erklärt, wie sich diese Masterarbeit in verwandten Arbeiten einbettet. \\

Die Masterarbeit versucht zwei der Wissenlücken zu füllen, die \textbf{Lutz Prechelt} 2019 in seiner Arbeit \textit{"Four presumed gaps in the software engineering research community's knowledge"} herausgestellt hat und zu denen er vorschlägt zu forschen. \cite{prechelt2019}\\
Mittels semi-strukturierten Interviews, Literaturrecherche und einen Beweis-orientierten Ansatz, wurden vier Wissenslücken aufgedeckt und vier Ziele für weitere Forschungen gesetzt. Für diese Masterarbeit ist zum Einem das Ziel interessant, dass Annahmen, welche den Umfang und die Relevanz einer Arbeit beschreiben, evaluiert werden, um einschätzen zu können, in welchen Kontexten eine Arbeit relevant ist. Zum Anderem ist das Ziel eine Taxonomie zu definieren, um eine einheitliche weitreichende Terminologie zu schaffen, das Thema dieser Masterarbeit. \\

 
\textbf{Andrew Forward} und \textbf{Timothy C. Lethbridge} haben 2008 in ihrer Arbeit \textit{"A Taxonomy of Software Types to Facilitate Search and
Evidence-Based Software Engineering"} eine Taxonomie zu verschiedenen Typen von Software erstellt um empirische Beweise, die auf Software basieren, zuordnen zu können.  \cite{taxonomy_software_types}. Dazu haben sie bereits bestehende Taxonomien zu Software Typen analysiert und durch Brainstorming eine eigene Taxonomie erstellt.\\
Forward und Lethbridge meinen, dass der Softwaretyp einen Kontext darstellt. Doch in dieser Arbeit bezieht sich ein Kontext einer empirischen Arbeit auf mehr als den Typ der Software, die untersucht oder vorgestellt wurde. Der Typ der Software ist lediglich eine Faktorenklasse, die Kontextfaktoren bezüglich des Softwaretypes zusammenfasst und mit weiteren Faktorenklassen einen Kontext abbilden kann.\\ 
Jedoch ist diese Arbeit trotzdem für die Masterarbeit von Nutzen, um weitere Schritte hinsichtlich einer umfassenderen Taxonomie zu tun. \\

In der Arbeit \textit{"Context in Industrial Software Engineering Research"} 2009 verfasst von \textbf{Kai Petersen} und \textbf{Claes Wohlin}, wurde eine Checkliste erstellt, um Wissenschaftler und Wissenschaftlerinnen zu unterstützen den Kontext ihrer Arbeiten zu beschreiben. \cite{contextIndustrialSE} Des Weiteren haben Petersen und Wohlin Literatur der Fachzeitschrift \textit{empirical software engineering} (EMSE) gesichtet, um zu evaluieren in wie weit Autoren und Autorinnen ihre Checkliste erfüllen. \\
Die Checkliste wurde unterteilt in sechs Kategorien, welche in dieser Masterarbeit als Faktorenklassen abgebildet werden könnten. Wie die Arbeit von Forward und Lethbridge sollte die Arbeit bei der Anfertigung der Taxonomie berücksichtigt werden. \\

\textbf{Tore Dyb\r{a}}, \textbf{Dag I.K. Sj\o{}berg} und \textbf{Daniela S. Cruzes} verfassten 2012 ebenfalls eine Arbeit zu dem Thema Kontexte in Software Engineering: \textit{"What works for whom, where, when, and why?: on the role of context in empirical software engineering"}. \cite{contextEmpiSE} Sie behaupten, dass in der empirischen Forschung zu Software Engineering, eher nach universalen Beziehungen gesucht werden, die unabhängig von dem Kontext sind. Jedoch lassen sich einige Forschungsfragen in unterschiedlichen Kontexten unterschiedlich beantworten. Sie erklären, wie eine Forschungsfragen kontextuell gefragt werden kann und dass die Beschreibung von Kontextfaktoren wichtig ist, um Wissen über mehrere Studien hinweg aufzubauen. \\
Die Verfasser und Verfasserinnen beziehen sich auf die Definition des Begriffes Kontext von \textbf{Gary Johns}, der in seiner Arbeit \textit{"The essential impact of context on organizational behavior"} untersucht, welchen Einfluss ein Kontext hat und in welchen Dimensionen ein Kontext beschrieben werden kann. \cite{impactContext}  Zwar forscht Johns in dem Fachgebiet zum organisatorischem Verhalten, jedoch sind seine Erkenntnisse und Schlussforgerungen aufschlussreich für diese Masterarbeit. \\

% Chapter 2
% - Wie bettet es sich in andere Arbeiten ein? Was ist nicht das Problem? Was wird nicht gelöst mit dieser Arbeit?
%% - Stand der Kunst, Vergleichbare Arbeiten (right wissenschaftl. Literatur)
