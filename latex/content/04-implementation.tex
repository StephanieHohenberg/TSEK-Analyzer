\chapter{Durchführung}
\label{chapter:impl}
Im Folgendem wird die Realisierung des im vorherigen Kapitel beschriebenden Vorgehen dokumentiert. Dafür werden die gesichteten Publikationen analysiert und die entstandenen Artefakte beschrieben.

\section{Kontexte}
%% TODO \\

\subsection{Thema}
%% TODO \\

\subsection{Eingrenzung und Abgrenzung}
%% TODO \\

\clearpage
\subsection{Verallgemeinbarkeit}

Im Folgendem werden die Kontextfaktoren beschrieben,
die zu dem Zweck der Verallgemeinbarkeit zugeordnet wurden.
Des Weiteren wird untersucht, in wie weit Autoren und Autorinnen über die Verallgemeinbarkeit ihrer Annahmen diskutieren. \\

Vorrangig werden lediglich die 59 Publikationen untersucht,
die mit einen ACM Award für \textit{Distinguished Paper} oder \textit{Distinguished Artefact} ausgezeichnet wurden und/oder einen Badge erhalten haben \textit{ACM Artifacts Evaluated Reusable} oder \textit{ACM Artifacts Available}. Es wird angenommen, dass diese Publikationen eine höhere Qualität aufweisen und mehr auf die Verallgemeinbarkeit eingangen wird. \\

Wie bereits in Kapitel \ref{chapter:vorgehen} beschrieben, wird angenommen, dass die Kontexte mit dem Zweck der Verallgemeinbarkeit
an bestimmten Stellen in einer Publikation vorkommen. Daher wurden lediglich folgende Stellen berücksichtigt: \textit{Threats of Validity}, \textit{Conclusion}, \textit{Limitations}, \textit{Future Work} und \textit{Discussion}. \\

Innerhalb 34 Publikationen wurden 35 Kontextfaktoren identifiziert.
Davon beziehen sich 24 Kontextfaktoren auf die Verallgemeinbarkeit,
8 auf die Abgrenzung des Themas, zwei auf das Thema selbst und ein Kontextfaktor beschreibt Details der Publikation.
Die Kontextfaktoren bezüglich der Verallgemeinbarkeit wurden in der Tabelle \ref{table:kontexte-verallgemeinbarkeit} zusammengefasst und in vier Faktorengruppen klassifiziert: Anwendung, Zielgruppe Programmiersprachen und Testtypen.
\\


\begin{table}[h!]
    \begin{tabular}{ r | l | l  }
        ID & Kontextfaktor & Klassifikation \\
        \hline
        b535 & Crowdtesting Plattform & Zielsoftware\\
        b535 & funktionale Tests & Testtypen \\
        b535 & Usability Tests & Testtypen \\
        b535 & Security Tests & Testtypen \\
        b535 & Performanz Tests & Testtypen \\
        a435 & Opensource System & Zielsoftware \\
        a435 & kommerzielle Closedsource Systeme & Zielsoftware \\
        a949 & convolutional neural networks & Zielsoftware  \\
        b110 & TensorFlow, Keras, Pytorch und andere DL frameworks & Zielsoftware  \\
        b159 & Philly und andere DL Plattformen & Zielsoftware \\
        a691 & ... & Programmiersprachen \\
        a309 & industrielle Teams & Zielgruppe  \\
    \end{tabular}
    \caption{Kontexte bezüglich der Verallgemeinbarkeit}
    \label{table:kontexte-verallgemeinbarkeit}
\end{table}


\clearpage
In ungefährt 3/4 der Publikationen (72,8 \%) ist die Sektion \textit{Threats to Validity} vorhanden. In 41,8\% dieser Publikationen wird diese Sektion in interne, externe und construct Validität aufgeteilt und in 20,9\% in interne und externe Validität. Jedoch wird meistens mehr auf die interne Validität eingegangen und beschrieben welche Maßnahmen getroffen wurden, um diese zu stärken und um Fehler, Bias oder Effekte zu vermeiden. Die externe Validität wird meist nur knapp beschrieben und es werden sehr wenig Aussagen zur Verallgemeinbarkeit getätigt. \\
Es fällt auf, dass anstatt über die Verallgemeinbarkeit zu diskutieren, die Studie und das Vorgehen erneut beschrieben wird. In vielen Publikationen (39,5\%) wird auf die Evaluation verwiesen, wie auf die Popularität der genutzten Benchmark
\footnote{ "Our use of this well-known and widely-used benchmark of real-world defects aims to ensure our results generalize" - \textit{"Causal Testing: Understanding Defects’ Root Causes"} von Brittany Johnson, Yuriy Brun und Alexandra Meliou [a087]}
und die Auswahl der evaluierten Testsubjekte.
In 8 Publikationen (18,6 \%) wird auf die Diversität oder auf die Limitierung der Teilnehmenden verwiesen.
\footnote{ "To bolster these observations we subsequently used a more diverse interview sample, which included participants from both large and small employers." - \textit{"A Tale from the Trenches: Cognitive Biases and Software Development"} von Souti Chattopadhyay, Nicholas Nelson, Audrey Au, Natalia Morales, Christopher Sanchez, Rahul Pandita und Anita Sarma [a654]} \\

\begin{table}[h!]
    \begin{tabular}{ l | c }
        & \# \\
        \hline
        Aufteilung in internal, external und construct validity & 18 \\
        Aufteilung in internal und external validity & 9 \\
        Beschreibung wie die interne Validität bestärkt wird & 7 \\
        \hline
        Sektion: \textit{Threats to Validity} & 43 \\
        Keine Sektion \textit{Threats to Validity} vorhanden & 16 \\
        \hline
        Limitations & 15 \\
        Future Work & 24 \\
        \hline
        Kein Fokus auf Verallgemeinbarkeit & 2 \\
        Keine Stellungnahme zur Verallgemeinbarkeit & 11 \\
        Behauptung: erstmalige Studie & 26 \\
        \hline
        Verweis auf der Diversität / Limitierung der Teilnehmenden & 8 \\
        Verweis auf andere Studien (scope, scale) & 2 \\
        Verweis auf eigenen Scope, Scale & 7 \\
        Verweis auf Vorgehen, Beschreibung der Studie & 7 \\
        Verweis auf Evaluation & 17 \\
        Verweis auf Implementierung & 3 \\
    \end{tabular}
    \caption{Analyse der Diskussion zur Verallgemeinbarkeit}
    \label{table:threats-of-validity}
\end{table}

%13 AWARDED PAPERS
%k - b535 - Beschreibung der Studie, future work
%_ - a087 - external, internal, construct validity, limitations, future work, how to mitigate, participants
%x - a654 - participants, how to mitigate internal treats, no focus on generalizability,
%k,x - a691 - limitations, no focus on generalizability
%_ - b073 nur Limitations, sonst nada
%x - a322 - nada, future work
%k - a309 - future work, limitations, ex/i/c validity, how to mitigate internal treats
%_ - a481 - how to mitigate internal treats, no external
%_ - b171 - ex/i/c validity, how to mitigate internal treats, participants
%k - b435 - ex/i/c validity
%k - a949 - limitations
%k - b110 - internal/external validity
%_ - b159 - discussion about generality


\clearpage

In zwei Publikationen wird ausgesagt, dass der Fokus der Studie nicht auf die Verallgemeinbarkeit liegt
\footnote{ "While desirable, generalizability was not the main focus of this study" - \textit{"A Tale from the Trenches: Cognitive Biases and Software Development"} von Souti Chattopadhyay, Nicholas Nelson, Audrey Au, Natalia Morales, Christopher Sanchez, Rahul Pandita und Anita Sarma [a654]}
und in 11 Publikationen (18,6\%) wird ausgesagt, dass die Ergebnisse möglicherweise nicht generalisierbar sind.
\footnote{ \textit{"Our findings might not be generalizable to other systems." - "Towards the Use of the Readily Available Tests from the Release
Pipeline as Performance Tests. Are We There Yet?"} von Zishuo Ding, Jinfu Chen und Weiyi Shang [b435] }
In diesen Aussagen wurden zwar Kontextfaktoren gefunden bezüglich der möglichen Verallgemeinbarkeit. Jedoch werden wenig Aussagen über diese Kontexte gemacht und außerdem wird nicht über die Verallgemeinbarkeit der Ergebnisse auf diese Kontexte argumentiert. \\

Des Weiteren fällt auf, dass in mehr als 1/4 der Publikation (27,1 \%) die Sektion \textit{Threats to Validity} nicht vorhanden ist. Da Kontextfaktoren auch in anderen Sektionen vorkommen können, wurde untersucht, ob neben der \textit{Conclusion}, Sektionen vorhanden sind, um die Begrenzungen der Arbeit und die zukünftige Arbeit zu erläutern.
In 10 der 16 der Arbeiten (62,5\%) sind solche Sektionen oder Passagen vorhanden. Im Vergleich mit den Publikationen mit \textit{Threats to Validity}, welcher in Tabelle \ref{table:sektionen} dargestellt ist, ist ein Unterschied in der Häufigkeit der Sektion \textit{Limitations} bemerkbar.
Es kann daher geschlussfolgert werden, dass in Publikationen ohne \textit{Threats to Validity} mehr über die Begrenzungen der Arbeit diskutiert wird, statt über die Validität und die Verallgemeinbarkeit.

\begin{table}[h!]
    \begin{tabular}{ r | l | l | l  | l }
        & \# & \textit{Limitations} & \textit{Future Work} & L + F   \\
        \hline
        \textit{Threats to Validity} vorhanden & 43 & 3 (6,9\%) & 12 (27,9\%) & 6 (13,95\%) \\
        \textit{Threats to Validity} nicht vorhanden & 16 & 4 (25\%) & 4 (25\%)& 2 (12,5\%)\\
        insgesamt & 59 & 7 (11,8\%) & 16 (27,1\%) & 8 (13,5\%)\\
    \end{tabular}
    \caption{Analyse der Vorkommen von Sektionen zu \textit{Threats to Validity, Limitations} (L) und \textit{Future Work} (F)}
    \label{table:sektionen}
\end{table}


Es ergibt sich eine Schnittmenge von 27 Publikationen (45,7\%), in denen nicht oder nur oberflächlich auf die Verallgemeinbarkeit eingegangen wird, sodass Lücken in der Diskussion der Verallgemeinbarkeit bestehen.
Lediglich in 25 Publikationen (42.3\%) wurden Kontexte bezüglich der Verallgemeinbarkeit gefunden.



\clearpage

\subsection{Details}
%% TODO \\

\section{Taxonomie}
%% TODO \\

% Chapter 4:
% - Dokumentation der Durchführung und der entstandenen Artefakte
% - Beschreibung besonderer Schwierigkeiten und wie sie gelöst, umgangen oder vermieden wurden (oder warum nicht)
% -	Software zur qualitativen Datenanalyse zum Sortieren, Statistik, Datenvisualisierung
